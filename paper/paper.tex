\documentclass[preprint,11pt]{article} \usepackage{amsmath}
\usepackage{listings} \usepackage{microtype} \usepackage[T1]{fontenc}

\title{Stellar Consensus Protocol Implementation} \author{Jeremy
  Rubin\\ \texttt{jr@mit.edu} \and John
  Holliman\\ \texttt{holliman@mit.edu} }
\begin{document}
\maketitle \special{papersize=8.5in,11in}
\setlength{\pdfpageheight}{\paperheight}
\setlength{\pdfpagewidth}{\paperwidth}

\section{Introduction}
Cryptocurrencies are systems which facilitate the execution of
payments, contracts, and other types of transactions over the internet
in secure and robust ways. There is a rich early history of
cryptocurrencies, ranging from Digi-Cash\cite{digicash} to
Peppercoin\cite{peppercoin}. These systems require trusted third
parties,
however, and ultimately were not successful. Bitcoin solved the
problem of reliance on trusted third parties through its Byzantine
Fault Tolerant consensus mechanism -- a proof of work based
blockchain.

Proof of work is a very costly and energy inefficient means to reaching
consensus as it requires solving dificult problems which wastes
computational resources.  Furthermore, transactions take a long time
to confirm and security is in question if a single group at any point
accounts for more than 50\% of the computing power.

The Stellar Consensus Protocol\cite{stellar} is a design of a Federated Byzantine
Agreement System, or a consensus protocol which relies on Federated
Voting for security rather than proof-of-work. This is a much more
efficient means of reaching consensus compared to proof-of-work,
although the lower expense might mean that the incentives to keep it
are lower.



\section{Stellar Consensus Protocol}
The Stellar Consensus Protocol is a four-phase Paxos-like consensus
protocol. Nodes exchange a series of ballots to vote to confirm, then
accept values. The protocol can be considered in a single instance
(i.e., determines one value) context, but it can also be easily extended
to multiple instance log replication. The Stellar Consensus protocol
is rather complex. As a result, a large part of the effort for this project was in
understanding the subtlies of how it works. Here we give a
best-effort summary on how it works, but for more understanding the
full paper\cite{stellar} is necessary. For simplicity, we will talk
about how it works in the context of a single slot.

Stellar Consensus Protocol hinges on a property called quorum
intersection.  Quorum intersection is the notion that all properly
behaving nodes have connections.  This property is, more formally, the
notion that you can remove some set of misbehaving nodes and nodes
which are dependent on them, and if quorum intersection holds the
nodes will not be partitioned. TODO....



\section{Design Overview}
We implemented the Stellar Consensus Protocol from a clean slate, not
referencing the existing implementation.
\subsection{C++}
We decided to implement our project in C++. Although neither of us
knew C++ well a priori, we considered it an important goal for out
project given that almost all cryptocurrency/consensus systems code is
written in C++. There were several stumbling blocks to get over, but
we are both know proficient in C++, which we consider to be a very
large reward of this project as we are now more comfortable
contributing code to existing implementations.
\subsection{Implementation Details}
\paragraph{Network}
We implemented a mock RPC interface. The reason we did this was so
that we could extend it to easily show certain byzantine conditions,
such as packet loss and reordering. We didn't implement a networked
RPC interface although our RPC abstract base class could be subclassed
to communicate over network.

Even though our network was not real, key functionality was not mocked
out.  For instance, we serialize and deserialize all messages to and
from JSON using the Cereal library. Node threads only communicate with
these mock network queues, there is no direct memory sharing.

\paragraph{Node}
Each node maintains a set of slots. Upon receipt of a message, the
node looks up the slot, creating it if it doesn't exist.  The node
then processes the message in the context of the slot. Slots do not
have an effect on one another.

\paragraph{Quorum}
The node maintains a quorum set of size $n$. It also chooses a
parameter $m < n$ of which any set of $m+1$ nodes constitutes a quorum
slice. Quorum selection is an open problem in Stellar Consensus
Protocol, it is unclear how to get nodes to select quorums such that
quorum intersection holds.

\subsection{Proof Of Timeout}
One open problem in Stellar Consensus is determining the mechanism by
which Nodes are allowed to propose arguments for the log. Stellar
consensus can be extremely inefficient in terms of number of messages
sent, especially with multiple proposers.

One solution which we added was adding a proof-of-work packet filter
for ballots. By requesting a solution to: $hash(value || slot ||
nonce) < bound$ with every ballot, two different values will take
different amounts of time to find solutions to which serves as a
randomized timeout which is valid in a byzantine context. This can
help the network make progress. The bound can be tuned based on
network activity. This also helps achieve anti-spam properties as
well.

Unlike in bitcoin, this proof-of-work is not directly incentivized,
therefore hopefully less prone to the development of ASIC hardware to
compute them. The only incentive is to spend funds more quickly.

\subsection{Applications}
We implemented two simple key value stores on top of the Stellar
Consensus Protocol: Asteroid and Comet.

\subsubsection*{Asteroid}

The Asteroid KV store has the following semantics:

\paragraph{Get(Key)} returns a pair of Version and Value. Gets are not put in the log, they are served
at whatever slot the server has read up to. Entries are individually
versioned in any case.

\paragraph{Put(Key, Value)}  reads the log like Get, and puts in an entry of Value under Key with Version + 1. Versions less than what is stored in the log will be ignored.

In a more full implementation, Key can be a public key, and value can
be a signed message by the key. This allows for the StellarKV to be
used a configuration updating consensus protocol. Other invariants
could be added per key perhaps as well. This could also serve as the
start of storing a balance as well for a transaction system. Keys
could also be modified to be hierarchal for better name spacing, e.g.
$Public Key||Pictures||10$.

Versioning allows for a performant way of de-duplicating entries which
get in the log at multiple slots. If a Version is younger than what is
in the log, the updates are not applied. Of course, a user should keep
track of the highest version they have sent and be sure that a server
reflects that change eventually.

\subsubsection*{Comet}

The Comet KV store was designed to mirror the 6.824 labs. Comet consists of
comet/client.cpp, comet/common.hpp, and comet/server.cpp.

Comet replicas stay identical unless some lag behind, in which case they 
are able to catch back up.

The Comet clients tries different replicas until one responds. Calls to Client.Get() and
Client.Put() appear to have affected all replicas in the same order and have
at-most-once semantics.

\subsection{Consensus Overview}
\bibliographystyle{abbrv}
\begingroup
\raggedright
\bibliography{paper}
\endgroup
\end{document}
